\documentclass[aspectratio=169]{beamer}

\usepackage{pgfpages}
%\setbeameroption{show notes on second screen=right}

\usepackage{url}
\usepackage{graphicx}
\usepackage{xspace}
\usepackage{subfigure}
\usepackage{amsmath}
\usepackage{amsfonts}
\usepackage{amssymb}
\usepackage{stmaryrd}
\usepackage{listings}
\usepackage{xcolor}
\usepackage{color}
\usepackage[many]{tcolorbox}

\input{listings-coq.tex}

\newcommand{\N}{\mathbb{N}}
\newcommand{\ltac}{\mbox{$\mathcal{L}_{tac}$}}
\newcommand{\HCOL}{\emph{HCOL}}
\newcommand{\SHCOL}{\texorpdfstring{$\Sigma$-\emph{HCOL}}{Sigma-HCOL}}
\newcommand{\DHCOL}{\emph{D-HCOL}}
\newcommand{\FHCOL}{\emph{F-HCOL}}
\newcommand{\IHCOL}{\emph{I-HCOL}}

\usetheme{Madrid}
\usecolortheme{scott}
\useoutertheme{scott}
\setbeamertemplate{navigation symbols}{}

\AtBeginSection[]{
  \begin{frame}
  \vfill
  \centering
  \begin{beamercolorbox}[sep=8pt,center,shadow=true,rounded=true]{title}
    \usebeamerfont{title}\insertsectionhead\par%
  \end{beamercolorbox}
  \vfill
  \end{frame}
}

\title{Reification of shallow-embedded DSLs in Coq with automated verification}

\author{Vadim Zaliva \textsuperscript{1},Matthieu Sozeau \textsuperscript{2}}
\institute[A]{\textsuperscript{1}Carnegie Mellon University\\ %
\textsuperscript{2}Inria \& IRIF, University Paris 7
}

\date{CoqPL'19}

\begin{document}

\begin{frame}
  \titlepage
\end{frame}

\section{Introduction}

\begin{frame}{Shallow and Deep Embedding}
  There are several ways to embed a domain specific language (DSL) in
  a host language (HL):
  \medskip
  \begin{description}
  \item[Deep] DSL's AST is represented as HL data structures and
    semantic is explicitly assigned to it.
  \item[Shallow] DSL is a subset of HL and semantic is inherited from HL.
  \end{description}

  \medskip
  The shallow embedding is excellent for quick prototyping, as it
  allows quick extension or modification of the embedded language.
  Meanwhile, deep embeddings are better suited for code transformation
  and compilation.
  
\end{frame}

\begin{frame}{Motivation: HELIX project}
  \begin{itemize}
  \item HELIX is program generation system which can
    generate high-performance code for a variety of linear algebra
    algorithms, such as discrete Fourier transform, discrete cosine
    transform, convolutions, and the discrete wavelet transform.
  \item HELIX is inspired by SPIRAL.
  \item Focuses on automatic translation a class of mathematical
    expressions to code.
  \item Revealing implicit iteration constructs and re-shaping them to
    match target platform parallelizm and vectorization capabilities. \note[item]{our focus on logical optimization vs. actual backend hardware
      code generators}
  \item Rigorously defined and formally verified.
  \item Implemented in Coq proof assistant.
  \item Allows non-linear operators.
  \item Presently, uses SPIRAL as an optimization oracle, but we verify
    it's findings.\note[item]{SPIRAL is system for find optimizations
      while HELIX is a system for reasoning about optimizations and
      proving their correctness}
  \item Uses LLVM as machine code generation backend.\note[item]{we
      decided it is easier to reason about LLVM semantics than about C semantics}
  \item Main application: Cyber-physical systems.\note[item]{CPS vs. DSP}
  \end{itemize}

  \note{
    This were we take off with HELIX. We forego matrix interpretation
    and treat this as a PL (Programming Language) where we reason
    about semantics, computation, compilation, and formal
    verification.
  }
\end{frame}

\begin{frame}{Motivation (contd.)}

  HELIX system uses transformation and translation steps involving
  several intermediate languages:
  
  \begin{figure}
    \centering
    \includegraphics[keepaspectratio=true,width=0.96\paperwidth]{figures/lang_seq_to_dhcol.eps}
  \end{figure}

  {\SHCOL} language is shallow embedded, while {\DHCOL} is deep
  embedded. We were looking for a translation technique between the
  two which is:
  \medskip
  \begin{enumerate}
  \item Automated - Can automatically translate arbitrary valid
    {\SHCOL} expressions.
  \item Verified - Provides semantic preservation guarantees.
  \end{enumerate}

  \note[item]{{\SHCOL} and {\HCOL} are operator languages given
    denotational semantics and shallow embedding is convenient 
    representation for reasoning about them.}
  \note[item]{Translation validation is used for {\SHCOL} and
    {\HCOL} verification, while we develop a cerfified compiler from
    {\DHCOL} to LLVM IR}
  \note[item]{{\DHCOL} has only operational semantics}

\end{frame}

\section{Example}

\begin{frame}[fragile]{An Example of Shallow Embedding in Coq}

  As an example let us consider a simple language of arithmetic
  expressions on natural numbers. It is shallow-embedded in Gallina
  and includes constants, bound variables, and three arithmetic
  operators: $+$, $-$, and $*$.
  
  \medskip
  Provided that $a, b, c, x \in \N$, the expression below is an
  example of a valid expression in the source language:
  \medskip
  \begin{lstlisting}[language=Coq, mathescape=true, basicstyle=\large,
    frame=single]
    2 + a*x*x + b*x + c.
  \end{lstlisting}

\end{frame}

\begin{frame}[fragile]{An Example of Deep Embedding in Coq}

  A deep-embedded variant of the same language includes the same
  operators but will be defined by an inductive type of its AST:

  \begin{lstlisting}[language=Coq, mathescape=true,
    frame=single, basicstyle=\footnotesize]
    Inductive NExpr: Type :=
    | NVar  : nat -> NExpr 
    | NConst: nat -> NExpr
    | NPlus : NExpr -> NExpr -> NExpr
    | NMinus: NExpr -> NExpr -> NExpr
    | NMult : NExpr -> NExpr -> NExpr.
  \end{lstlisting}

  Our sample expression, in deep-embedded target language, looks like:
  
  \begin{lstlisting}[language=Coq, mathescape=true,
    basicstyle=\footnotesize, frame=single]
    NPlus (NPlus (NPlus (NConst 2)
      (NMult (NMult (NVar 3) (NVar 0)) (NVar 0)))
        (NMult (NVar 2) (NVar 0))) (NVar 1)
  \end{lstlisting}
  
\end{frame}

\section{Template Coq}

\begin{frame}[fragile]{Template Coq}
  TemplateCoq is quoting library for Coq. It takes Coq terms and
  constructs a representation of their syntax tree as a Coq inductive
  data type:

\begin{lstlisting}[language=Coq, mathescape=true, frame=none, basicstyle=\scriptsize]
Inductive term : Set :=
| tRel       : nat -> term
| tVar       : ident -> term (* For free variables (e.g. in a goal) *)
| tMeta      : nat -> term 
| tEvar      : nat -> list term -> term
| tSort      : universe -> term
| tCast      : term -> cast_kind -> term -> term
| tProd      : name -> term (* the type *) -> term -> term
| tLambda    : name -> term (* the type *) -> term -> term
| tLetIn     : name -> term (* the term *) -> term (* the type *) -> term -> term
| tApp       : term -> list term -> term
| tConst     : kername -> universe_instance -> term
| tInd       : inductive -> universe_instance -> term
| tConstruct : inductive -> nat -> universe_instance -> term
| tCase      : (inductive * nat) (* # of parameters *) -> term (* type info *)
               -> term (* discriminee *) -> list (nat * term) (* branches *) -> term
| tProj      : projection -> term -> term
| tFix       : mfixpoint term -> nat -> term
| tCoFix     : mfixpoint term -> nat -> term.
  \end{lstlisting}

  \note[item]{This is how Coq AST tree type looks like. TemplateCoq
    ``quote'' operation reifies Coq terms to this type. The inverse
    operation is called ``unquoting''}
  \note[item]{The AST representation is based on the kernel’s term representation.}
\end{frame}

\begin{frame}[fragile]{The Template Monad}
\begin{lstlisting}[language=Coq, mathescape=true, frame=none, basicstyle=\tiny]
Cumulative Inductive TemplateMonad@{t u} : Type@{t} -> Type :=
(* Monadic operations *)
| tmReturn : forall {A:Type@{t}}, A -> TemplateMonad A
| tmBind : forall {A B : Type@{t}}, TemplateMonad A -> (A -> TemplateMonad B) -> TemplateMonad B
(* General commands *)
| tmPrint : forall {A:Type@{t}}, A -> TemplateMonad unit
| tmFail : forall {A:Type@{t}}, string -> TemplateMonad A
| tmEval : reductionStrategy -> forall {A:Type@{t}}, A -> TemplateMonad A
| tmDefinition : ident -> forall {A:Type@{t}}, A -> TemplateMonad A
| tmAxiom : ident -> forall A : Type@{t}, TemplateMonad A
| tmLemma : ident -> forall A : Type@{t}, TemplateMonad A
| tmFreshName : ident -> TemplateMonad ident
| tmAbout : ident -> TemplateMonad (option global_reference)
| tmCurrentModPath : unit -> TemplateMonad string
(* Quoting and unquoting commands *)
| tmQuote : forall {A:Type@{t}}, A  -> TemplateMonad Ast.term
| tmQuoteRec : forall {A:Type@{t}}, A  -> TemplateMonad program
| tmQuoteInductive : kername -> TemplateMonad mutual_inductive_body
| tmQuoteUniverses : unit -> TemplateMonad uGraph.t
| tmQuoteConstant : kername -> bool (* bypass opacity? *) -> TemplateMonad constant_entry
| tmMkDefinition : ident -> Ast.term -> TemplateMonad unit
| tmMkInductive : mutual_inductive_entry -> TemplateMonad unit
| tmUnquote : Ast.term  -> TemplateMonad typed_term@{u}
| tmUnquoteTyped : forall A : Type@{t}, Ast.term -> TemplateMonad A
(* Typeclass registration and querying for an instance *)
| tmExistingInstance : ident -> TemplateMonad unit
| tmInferInstance : forall A : Type@{t}, TemplateMonad (option A)
.
\end{lstlisting}

\note[item]{While TemplateCoq provides quoting and unquoting
  vernacular, we are mostly interested in programming interface
  represented by Template Monad (shown here).}

\note[item]{It allows to write \textit{template programs} which
  inspect (via quoting) as well as construct new terms. In particualar
  they can create new Definitions and Lemmas}.

\note[item]{A template program \lstinline[language=Coq]{t} of type
  \lstinline[language=Coq]{TemplateMonad A} could be run with
  \lstinline[language=Coq]{Run TemplateProgram t}.}
\end{frame}

\section{Approach}

\begin{frame}[fragile]{Reification Approach}
  
  \begin{itemize}
  \item Implemented as \textit{Template Program} named \lstinline[language=Coq]{reifyNExp}
  \item The input expressions are given in a closed form, where all
    variables first need to be introduced via lambda
    bindings. E.g. $\mathbb{N}$ or
    $\N \rightarrow \ldots \rightarrow \N$.
  \item \lstinline[language=Coq]{reifyNExp} also takes two names (as strings). The first
    is used as the name of a new definition, to which the expression
    in the target language will be bound. The second is the name to
    which the semantic preservation lemma will be bound.
  \end{itemize}

  \begin{lstlisting}[language=Coq, mathescape=true, frame=single]
  Polymorphic Definition reifyNExp@{t u}
    {A:Type@{t}} (res_name lemma_name: string) (nexpr:A)
    : TemplateMonad@{t u} unit.
  \end{lstlisting}

  When executed, if successful, \lstinline[language=Coq]{reifyNExp} will create a new
  definition and new lemma with the given names. If not, it will fail
  with an error. \note{The reason for a failure might be that the
    expression contains constructs which are legal in Gallina but not
    part of our embedded language.}

\end{frame}


\begin{frame}[fragile]{Reification Implementation}
  
  \begin{itemize}
  \item All variables are converted into parameters via \lstinline[language=Coq]{forall}.
  \item Gallina AST (as \textit{quoted} by TemplateCoq) of the
    original expression is traversed and an expression in target
    language is generated.
  \item Variable names are converted to de Bruijn indices.
  \end{itemize}
  \medskip
  Our sample expression will be compiled to the following
  deep-embedded form:
  
    \begin{lstlisting}[language=Coq, mathescape=true,frame=single]
    NPlus (NPlus (NPlus (NConst 2)
        (NMult (NMult (NVar 3) (NVar 0)) (NVar 0)))
        (NMult (NVar 2) (NVar 0))) (NVar 1)
  \end{lstlisting}

\end{frame}

\begin{frame}[fragile]{Semantic Preservation Approach}
  
  Semantics of our deep embedded language is defined by evaluation
  function:

  \begin{lstlisting}[language=Coq, mathescape=true,frame=single]
    Definition evalContext:Type := list nat.
    Fixpoint evalNexp ($\Gamma$:evalContext) (e:NExpr): option nat.
  \end{lstlisting}

  The semantic preservation is expressed as a heterogeneous relation
  between expressions in the source and target languages:

  \begin{lstlisting}[language=Coq, mathescape=true,frame=single]
    Definition NExpr_term_equiv ($\Gamma$: evalContext)
    (d: NExpr) (s: nat) : Prop := evalNexp $\Gamma$ d = Some s.
  \end{lstlisting}

  For our sample expression the following lemma will be generated:
  
  \begin{lstlisting}[language=Coq, mathescape=true,frame=single]
forall x c b a : nat, NExpr_term_equiv [x; c; b; a]
    NPlus (NPlus (NPlus (NConst 2)
        (NMult (NMult (NVar 3) (NVar 0)) (NVar 0)))
        (NMult (NVar 2) (NVar 0))) (NVar 1)
    (2 + a * x * x + b * x + c)
  \end{lstlisting}
  
  \note[item]{The evaluation may fail, for example, if the expression references a
    variable not present in the evaluation context.}
  \note[item]{All variables are converted into parameters via \lstinline[language=Coq]{forall}}
\end{frame}

\begin{frame}[fragile]{Automated Proof of Semantic Preservation Lemma}

  The general idea is to define and prove semantic
  equivalence lemmas for each pair of operators. For example:

  \begin{lstlisting}[language=Coq, mathescape=true,frame=single, basicstyle=\footnotesize]
Lemma NExpr_mul_equiv ($\Gamma$: evalContext) {a b a' b' }:
  NExpr_term_equiv $\Gamma$ a a' -> NExpr_term_equiv $\Gamma$ b b' ->
  NExpr_term_equiv $\Gamma$ (NMult a b) (Nat.mul a' b').
\end{lstlisting}

Because the expressions in the original and target languages have the
same structure, such proof can be automated, for example by use of
{eauto} tactic.
  
  \begin{figure}[h]
    \includegraphics[width=0.5\columnwidth]{figures/trees.eps}
  \end{figure}

  \note{Compositional proof}
\end{frame}


\section{Case study}

\begin{frame}{Practical use in HELIX system}

  The input is shallow embedded {\SHCOL} language:
  \begin{itemize}
  \item An operator language on finite-dimensional vectors.
  \item Includes meta-operators\note[item]{meta-operators are just
      higher order functions}
  \item Includes binders.
  \item Operates on sparse vectors with optional collistion tracking.
  \item Uses logical parameters (of type \lstinline[language=Coq]{Prop}).
  \end{itemize}
  \medskip
  The output is deep embedded {\DHCOL} language:
  \begin{itemize}
  \item No logical parameters.
  \item Using de Bruijn indices for variables.
  \item Only a subset of {\SHCOL} operators are
    represented.\note[item]{{\SHCOL} is normalized before reification
      and some operators are eliminated}
  \item Operates on dense vectors.
  \item A fixed set of \textit{intirinsic} functions (e.g. $+,-,max$) is defined.
  \end{itemize}
  
\end{frame}

\begin{frame}[fragile]{{\SHCOL} Example}
  \vspace{-2mm}
  \lstinputlisting[language=Coq,
  mathescape=true,
  columns=fullflexible,
  frame=none,
  basicstyle=\scriptsize]
  {dynwin_SHCOL1.txt}
  \note[item]{You can see some function composition, lambdas,
    algebraic expressions, Props}
  \note[item]{$a$ is global parameter}
  \note[item]{Dependent types (e.g. vectors, operators)}
\end{frame}

\begin{frame}[fragile]{{\DHCOL} Example}
  \vspace{-2mm}
  \lstinputlisting[language=Coq,
  mathescape=true,
  columns=fullflexible,
  frame=none,
  basicstyle=\scriptsize]
  {dynwin_DHCOL1.txt}
  \note[item]{Implicit binders, variables via de Bruijn indices}
  \note[item]{Dependent types (e.g. vectors, operators)}
\end{frame}

\begin{frame}[fragile]{Reifying dependently typed operators}
  {}{\DHCOL} operator type family is indexed by dimensions of input
  and output vectors:
  \begin{lstlisting}[language=Coq, mathescape=true, frame=single]
  Inductive DSHOperator: nat -> nat -> Type
  \end{lstlisting}

  To deal with it we defined a record and cast function:

  \begin{lstlisting}[language=Coq, mathescape=true, frame=single]
  Record reifyResult := { rei_i: nat; rei_o: nat; rei_op: DSHOperator rei_i rei_o; }.
  Definition castReifyResult (i o:nat) (rr:reifyResult): TemplateMonad (DSHOperator i o).
  \end{lstlisting}

  Our reification templete program returns
  \lstinline[language=Coq]{TemplateMonad reifyResult}. The reification
  of a function composition is implemented as:

  \begin{lstlisting}[language=Coq, mathescape=true, frame=single, basicstyle=\scriptsize]
  | Some n_SHCompose, [fm ; i1 ; o2 ; o3 ; op1 ; op2] =>
    ni1 <- tmUnquoteTyped nat$\,$i1 ;; no2 <- tmUnquoteTyped nat$\,$o2 ;; no3 <- tmUnquoteTyped nat$\,$o3 ;;
    cop1' <- compileSHCOL vars op1 ;; cop2' <- compileSHCOL vars op2 ;;
    let '(_, _, cop1) := (cop1':varbindings*term*reifyResult) in
    let '(_, _, cop2) := (cop2':varbindings*term*reifyResult) in
    cop1 <- castReifyResult no2 no3 cop1 ;; cop2 <- castReifyResult ni1 no2 cop2 ;;
         tmReturn (vars, fm, {| rei_i:=ni1; rei_o:=no3; rei_op:=@DSHCompose ni1 no2 no3 cop1 cop2 |})
  \end{lstlisting}
    
\end{frame}

\begin{frame}[fragile]{\emph{coq-switch} plugin}
  While parsing Coq's AST we have to do a lot of string matching, like
  this:
  \begin{lstlisting}[language=Coq, mathescape=true, frame=single,
    basicstyle=\scriptsize]
Definition parse_SHCOL_Op_Name (s:string): option SHCOL_Op_Names :=
  if string_dec s "Helix.SigmaHCOL.SigmaHCOL.eUnion" then Some n_eUnion
  else if string_dec s "Helix.SigmaHCOL.SigmaHCOL.eT" then Some n_eT
    $\ldots$
  else None.    
\end{lstlisting}

Writing manually such biolerplate code is tiresome and error-prone. We
developed \emph{coq-switch} plugin to to automate such tasks, for any
decidable type:

\begin{lstlisting}[language=Coq, mathescape=true, frame=single, basicstyle=\scriptsize]
  Run TemplateProgram
    (mkSwitch string string_beq
              [("Helix.SigmaHCOL.SigmaHCOL.eUnion", "n_eUnion") ;
               ("Helix.SigmaHCOL.SigmaHCOL.eT"    , "n_eT"    ) ;
                 $\ldots$]
              "SHCOL_Op_Names" "parse_SHCOL_Op_Name"
    ).
\end{lstlisting}

It will generate inductive type \lstinline[language=Coq]{SHCOL_Op_Names}
with constructors \lstinline[language=Coq]{n_eUnion}, \lstinline[language=Coq]{n_eT $\ldots$} and a
function: \lstinline[language=Coq]{parse_SHCOL_Op_Name: string -> option SHCOL_Op_Names}.

\note[item]{Aviablable via OPAM.}
\end{frame}

\section{Conclusions}

\begin{frame}{{\SHCOL} to {\DHCOL} reification in numbers}
  \begin{itemize}
  \item Development time: \textasciitilde 2 weeks
  \item Lines on code:
    \begin{description}
    \item[Spec] 601
    \item[Proofs] 830
    \end{description}
  \item Execution time (for simple expression containing 24 {\SHCOL} operators):
    \begin{description}
    \item[Reification] 4s
    \item[Proofs] 1s
    \end{description}
\end{itemize}
  

  \note[1]{developement time includes experiments, finding and refining right approach}
\end{frame}
    
\begin{frame}{Questions?}

  Links:
  
  \begin{itemize}
  \item Full example: \url{https://github.com/vzaliva/CoqPL19-paper}
  \item TemplateCoq:
    \url{https://metacoq.github.io/metacoq/}
  \item \emph{coq-switch} plugin \url{https://github.com/vzaliva/coq-switch}
  \item \textit{``HELIX: a case
    study of a formal verification of high performance program
    generation.''}, Vadim Zaliva, Franz Franchetti. FHPC 2018. 
  \end{itemize}

  Contact:

  \begin{itemize}
  \item Vadim Zaliva: \url{http://www.crocodile.org/lord/}
  \item Matthieu Sozeau: \url{https://www.irif.fr/~sozeau/}
  \end{itemize}  
  
\end{frame}

\section{Backup slides}

\begin{frame}{From Denotational to Operational Semantics}
  \begin{center}
    \includegraphics[width=0.8\columnwidth]{figures/semantics.eps}
  \end{center}
  
  \begin{enumerate}
  \item {\SHCOL} has both denotational and operational semantics.
  \item Structural properties allow to switch from one to another.
  \end{enumerate}


  \note{this is a ``big picture'' slide from HELIX presentation
    explaining where reification step fits and how different semantics relate}
\end{frame}


\end{document}


