%% For double-blind review submission, w/o CCS and ACM Reference (max submission space)
\documentclass[sigplan]{acmart}\settopmatter{printfolios=true,printccs=false,printacmref=false}
%% For double-blind review submission, w/ CCS and ACM Reference
%\documentclass[acmsmall,review,anonymous]{acmart}\settopmatter{printfolios=true}
%% For single-blind review submission, w/o CCS and ACM Reference (max submission space)
%\documentclass[acmsmall,review]{acmart}\settopmatter{printfolios=true,printccs=false,printacmref=false}
%% For single-blind review submission, w/ CCS and ACM Reference
%\documentclass[acmsmall,review]{acmart}\settopmatter{printfolios=true}
%% For final camera-ready submission, w/ required CCS and ACM Reference
%\documentclass[acmsmall]{acmart}\settopmatter{}


%% Journal information
%% Supplied to authors by publisher for camera-ready submission;
%% use defaults for review submission.
\acmJournal{PACMPL}
\acmVolume{1}
\acmNumber{POPL} % CONF = POPL or ICFP or OOPSLA
\acmArticle{1}
\acmYear{2019}
\acmMonth{1}
\acmDOI{} % \acmDOI{10.1145/nnnnnnn.nnnnnnn}
\startPage{1}

%% Copyright information
%% Supplied to authors (based on authors' rights management selection;
%% see authors.acm.org) by publisher for camera-ready submission;
%% use 'none' for review submission.
\setcopyright{none}
%\setcopyright{acmcopyright}
%\setcopyright{acmlicensed}
%\setcopyright{rightsretained}
%\copyrightyear{2018}           %% If different from \acmYear

%% Bibliography style
\bibliographystyle{ACM-Reference-Format}
%% Citation style
%% Note: author/year citations are required for papers published as an
%% issue of PACMPL.
\citestyle{acmauthoryear}   %% For author/year citations


%% Some recommended packages.
\usepackage{booktabs}   %% For formal tables:
                        %% http://ctan.org/pkg/booktabs
\usepackage{subcaption} %% For complex figures with subfigures/subcaptions
                        %% http://ctan.org/pkg/subcaption


\begin{document}

%% Title information
\title{Reification of shallow-embedded DSL in Coq with automated verification}         %% [Short Title] is optional;
                                        %% when present, will be used in
                                        %% header instead of Full Title.
%\titlenote{with title note}             %% \titlenote is optional;
                                        %% can be repeated if necessary;
                                        %% contents suppressed with 'anonymous'
%\subtitle{Subtitle}                     %% \subtitle is optional
%\subtitlenote{with subtitle note}       %% \subtitlenote is optional;
                                        %% can be repeated if necessary;
                                        %% contents suppressed with 'anonymous'


\author{Vadim Zaliva}
\affiliation{
  \institution{Carnegie Mellon University}
}
\email{vzaliva@cmu.edu}

\author{Matthieu Sozeau}
\affiliation{
  \institution{INRIA}
}
\email{mattam@mattam.org}


\begin{abstract}
  Shallow and deep embeddings have their pros and cons.  For example
  shallow embedding is excellent for quick prototyping, while deep
  embedding is better suited for code transformation and compilation.
  Thus it might be useful to be able to switch from shallow to deep
  embedding while making sure the semantic of embedded language is
  preserved. We will demonstrate a working approach for implementing
  and proving such conversion using TemplateCoq.
\end{abstract}


%% 2012 ACM Computing Classification System (CSS) concepts
%% Generate at 'http://dl.acm.org/ccs/ccs.cfm'.
\begin{CCSXML}
<ccs2012>
<concept>
<concept_id>10011007.10011006.10011008</concept_id>
<concept_desc>Software and its engineering~General programming languages</concept_desc>
<concept_significance>500</concept_significance>
</concept>
<concept>
<concept_id>10003456.10003457.10003521.10003525</concept_id>
<concept_desc>Social and professional topics~History of programming languages</concept_desc>
<concept_significance>300</concept_significance>
</concept>
</ccs2012>
\end{CCSXML}

\ccsdesc[500]{Software and its engineering~General programming languages}
\ccsdesc[300]{Social and professional topics~History of programming languages}
%% End of generated code


%% Keywords
%% comma separated list
\keywords{coq}  %% \keywords are mandatory in final camera-ready submission


%% \maketitle
%% Note: \maketitle command must come after title commands, author
%% commands, abstract environment, Computing Classification System
%% environment and commands, and keywords command.
\maketitle

\section{Introduction}

In the course of our work on HELIX system\cite{helixFHPC18} we faced
the problem of writing a certified compiled for domain specific
language $\Sigma$-HCOL shallow-embedded in Coq. The approach presented
in this report is the result of summer 2018 collaboration visit to
INRIA where with TemplateCoq\cite{anand2018towards} first suggested by
Matthieu Sozeau and successfully implemented.


\section{Translation}


\subsection{Misc}
\begin{itemize}
\item Stripping Props
\item Target language could be a subset of original
\item From variable names to DeBruijn indices
\item Dealing with dependent types (\verb|TemplateMonad reifyResult| in HELIX)
\end{itemize}
\section{Semantics Preservation}
  
\section{Future and Related work}
\begin{itemize}
\item From deep to shallow?
\item Generalizing as a framework
\end{itemize}

%% Bibliography
%\bibliography{bibfile}


%% Appendix
%\appendix
%\section{Appendix}


\nocite{*}
\bibliographystyle{acm}
\bibliography{paper}

\end{document}
